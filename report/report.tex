% !TEX TS-program = pdflatex
% !TEX encoding = UTF-8 Unicode

% This is a simple template for a LaTeX document using the "article" class.
% See "book", "report", "letter" for other types of document.

\documentclass[11pt]{article} % use larger type; default would be 10pt

\usepackage[utf8]{inputenc} % set input encoding (not needed with XeLaTeX)

%%% Examples of Article customizations
% These packages are optional, depending whether you want the features they provide.
% See the LaTeX Companion or other references for full information.

%%% PAGE DIMENSIONS
\usepackage{geometry} % to change the page dimensions
\geometry{a4paper} % or letterpaper (US) or a5paper or....
% \geometry{margin=2in} % for example, change the margins to 2 inches all round
% \geometry{landscape} % set up the page for landscape
%   read geometry.pdf for detailed page layout information

\usepackage{graphicx} % support the \includegraphics command and options

% \usepackage[parfill]{parskip} % Activate to begin paragraphs with an empty line rather than an indent

%%% PACKAGES
\usepackage{booktabs} % for much better looking tables
\usepackage{array} % for better arrays (eg matrices) in maths
\usepackage{paralist} % very flexible & customisable lists (eg. enumerate/itemize, etc.)
\usepackage{verbatim} % adds environment for commenting out blocks of text & for better verbatim
\usepackage{subfig} % make it possible to include more than one captioned figure/table in a single float
% These packages are all incorporated in the memoir class to one degree or another...

\usepackage{amsmath}
\usepackage{algorithm}
\usepackage{algpseudocode}
\usepackage{tikz-qtree}
\usepackage{enumitem}

%%% HEADERS & FOOTERS
\usepackage{fancyhdr} % This should be set AFTER setting up the page geometry
\pagestyle{fancy} % options: empty , plain , fancy
\renewcommand{\headrulewidth}{0pt} % customise the layout...
\lhead{}\chead{}\rhead{}
\lfoot{}\cfoot{\thepage}\rfoot{}

%%% SECTION TITLE APPEARANCE
\usepackage{sectsty}
\allsectionsfont{\sffamily\mdseries\upshape} % (See the fntguide.pdf for font help)
% (This matches ConTeXt defaults)

%%% ToC (table of contents) APPEARANCE
\usepackage[nottoc,notlof,notlot]{tocbibind} % Put the bibliography in the ToC
\usepackage[titles,subfigure]{tocloft} % Alter the style of the Table of Contents
\renewcommand{\cftsecfont}{\rmfamily\mdseries\upshape}
\renewcommand{\cftsecpagefont}{\rmfamily\mdseries\upshape} % No bold!

%%% END Article customizations

\title{Distributed Computing}
\author{ffgt86}
%\date{} % Activate to display a given date or no date (if empty),
         % otherwise the current date is printed 

\begin{document}
\maketitle

\section{Perform a precise analysis of the time complexity of the Flooding algorithm.}

Let $G = (V_G, E_G)$ be a connected topology graph with specified root $p_r$.  $p_r$ has message $\langle M \rangle$ at $t = 0$.  Let a \textit{neighbour} of a vertex $v$ be a vertex $u$ such that $(u, v) \in E_G$. 
Let $dist(p_i, p_j)$ denote the distance between $p_i, p_j \in G$. Let $D = max\{dist(p_i, p_j) \ | \ (p_i, p_j) \in V_G \}$ be the diameter of $G$. 

\subsection{The synchronous model}

Let $A$ be the synchronous Flooding algorithm. Analysis is by induction.\\

\noindent
\textit{Base case}: each neighbour of $p_r$ receives $\langle M \rangle$ in the first round, $t = 1$.\\
\textit{Step case}: a neighbour $u$ of a node $v$ receives $\langle M \rangle$ at $t$ if $v$ received $\langle M \rangle$ at $t - 1$. True by description of $A$. \\

Each node at distance $t$ from $p_r$ therefore receives $\langle M \rangle$ in round $t$. $A$ terminates when every node has $\langle M \rangle$. The last node, at distance $D$ from $p_r$ will receive $\langle M \rangle$ at $t = D$. The time complexity of $A$ is therefore $O(D)$.

\subsection{The asynchronous model}

Let $B$ be the asynchronous Flooding algorithm. In the asynchronous model, time is expressed in terms of \textit{maximum message delays}, where a message delay is the time elapsed between the computation event that sends  $\langle M \rangle$ and the computation event that processes $\langle M \rangle$. This is set to $1$ for convenience. 

Therefore each message effectively \textit{is} transmitted in a round, as in the synchronous model, where each round lasts the maximum message delay, $1$. The time complexity of $B$ is therefore exactly the same as $A$ in $1.1$, by exactly the same proof: $O(D)$.

\section{Consider an anonymous ring where processors start with binary inputs.}

\subsection{Give an argument that there is no uniform synchronous algorithm for computing the AND of the input bits.}

Proof is by contradiction. Assume that there is an uniform synchronous algorithm $A$ that computes AND. Assume a ring where all inputs are $1$. In any round $i$ the states of all processors are identical, and these states do not depend on the size $n$ of the ring, as the algorithm is uniform. As $A$ is correct, there must exist a round $t$ such that all processors terminate and output $1$.

Now run $A$ on a ring of size $2(t + 1)$, with $1$ processor $p$ with input $0$ and $2t + 1$ processors with input $1$. $p$ disseminates $0$. However, the processor 'opposite' $p$ will not receive $0$ by round $t$; it will instead terminate and output $1$. This contradicts the assumption that $A$ is correct (in which case $0$ should be output at all processors).  

\subsection{Present an asynchronous (non-uniform) algorithm for computing the AND. The algorithm should send $O(n^2)$ messages in the worst-case.}

Let $n$ be the number of processors. Each processor sends a message to its right neighbour and waits for messages from its left neighbour. Each message $m$ comprises a counter $hop = 0$ and a state $x$. $hop$ is incremented every time $m$ is sent. If $m.hop == n$, the message has arrived where it started. 

When a processor $p$ with input state $x$ receives a message $m$, it updates $m.x = p.x \land m.x$. It then checks $m.hop$. If $m.hop < n$, it forwards the message to the right. If $m.hop == n$, the processor terminates. 

As each processor sends a message, and each message makes $n$ hops, the message complexity is $O(n^2)$.

\subsection{Present a synchronous algorithm for computing the AND. The algorithm should send $O(n)$ messages in the worst case.}

Let $n$ be the number of processors. A processor with an initial state of $0$ sends a message in both directions and halts (in state $0$). A processor with an initial state of $1$ waits for $\lfloor n / 2 \rfloor$ cycles. If it receives a message during that period, then it forwards it and halts in state $0$. If by cycle $\lfloor n / 2 \rfloor$ a processor has not received any message, it halts in state $1$. The total number of messages sent is $O(n)$.


\end{document}