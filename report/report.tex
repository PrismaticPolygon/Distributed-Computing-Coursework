% !TEX TS-program = pdflatex
% !TEX encoding = UTF-8 Unicode

% This is a simple template for a LaTeX document using the "article" class.
% See "book", "report", "letter" for other types of document.

\documentclass[11pt]{article} % use larger type; default would be 10pt

\usepackage[utf8]{inputenc} % set input encoding (not needed with XeLaTeX)

%%% Examples of Article customizations
% These packages are optional, depending whether you want the features they provide.
% See the LaTeX Companion or other references for full information.

%%% PAGE DIMENSIONS
\usepackage{geometry} % to change the page dimensions
\geometry{a4paper} % or letterpaper (US) or a5paper or....
% \geometry{margin=2in} % for example, change the margins to 2 inches all round
% \geometry{landscape} % set up the page for landscape
%   read geometry.pdf for detailed page layout information

\usepackage{graphicx} % support the \includegraphics command and options

% \usepackage[parfill]{parskip} % Activate to begin paragraphs with an empty line rather than an indent

%%% PACKAGES
\usepackage{booktabs} % for much better looking tables
\usepackage{array} % for better arrays (eg matrices) in maths
\usepackage{paralist} % very flexible & customisable lists (eg. enumerate/itemize, etc.)
\usepackage{verbatim} % adds environment for commenting out blocks of text & for better verbatim
\usepackage{subfig} % make it possible to include more than one captioned figure/table in a single float
% These packages are all incorporated in the memoir class to one degree or another...

\usepackage{amsmath}
\usepackage{algorithm}
\usepackage{algpseudocode}
\usepackage{tikz-qtree}
\usepackage{enumitem}

%%% HEADERS & FOOTERS
\usepackage{fancyhdr} % This should be set AFTER setting up the page geometry
\pagestyle{fancy} % options: empty , plain , fancy
\renewcommand{\headrulewidth}{0pt} % customise the layout...
\lhead{}\chead{}\rhead{}
\lfoot{}\cfoot{\thepage}\rfoot{}

%%% SECTION TITLE APPEARANCE
\usepackage{sectsty}
\allsectionsfont{\sffamily\mdseries\upshape} % (See the fntguide.pdf for font help)
% (This matches ConTeXt defaults)

%%% ToC (table of contents) APPEARANCE
\usepackage[nottoc,notlof,notlot]{tocbibind} % Put the bibliography in the ToC
\usepackage[titles,subfigure]{tocloft} % Alter the style of the Table of Contents
\renewcommand{\cftsecfont}{\rmfamily\mdseries\upshape}
\renewcommand{\cftsecpagefont}{\rmfamily\mdseries\upshape} % No bold!

%%% END Article customizations

\title{Distributed Computing}
\author{ffgt86}
%\date{} % Activate to display a given date or no date (if empty),
         % otherwise the current date is printed 

\begin{document}
\maketitle

\section{Perform a precise analysis of the time complexity of the Flooding algorithm.}
\subsection{The synchronous model}

In the synchronous model, process execution speeds and message delivery delays are upper-bounded by a fixed $k$.

Every processor receives $M$ after at most $D$ time and at most $|E|$ messages, where $D$ is the diameter of the network, and $E$ is the set of (directed) edges in the network. Proof is by induction.

Let $d(root, v) = k > 0$. Then $v$ has a neighbour $u$ such that $d(root, u) = k - 1$. By the induction hypothesis, $u$ receives $M$ for the first time no later than time $k - 1$. $u$ sends $M$ to all its neighbours, including $v$ at $k$, so $M$ arrives at $v$ no later than time $(k - 1) + 1 = k$.

Each process only sends $M$ to its neighbours once, so each edge carries at most one copy of $M$. The message complexity is therefore $|E$.

\subsection{The asynchronous model}

\section{Consider an anonymous ring where processors start with binary inputs.}

\subsection{Give an argument that there is no uniform synchronous algorithm for computing the AND of the input bits.}
\subsection{Present an asynchronous (non-uniform) algorithm for computing the AND. The algorithm should send $O(n^2)$ messages in the worst-case.}
\subsection{Present a synchronous algorithm for computing the AND. The algorithm should send $O(n)$ messages in the worst case.}

\end{document}